\documentclass[oneside]{article}

\usepackage{amsmath}
\usepackage{amssymb}
\usepackage{xurl}
\usepackage{hyperref}

\begin{document}
\title{How to Register WinRAR for a License File}
\author{GreenYun}
\maketitle

\section{Introduction}
WinRAR is a file archiver developed by Eugene Roshal.
Trail version of WinRAR can be download and distributed freely.
Sold and supported by win.rar GmbH, WinRAR is allowed to be used in a 40-day test period.
A license should be purchased to continue using the software.

After a person/organization/company pay to win.rar GmbH, one will receive a file named ``RARreg.key'' for registering the software.
The file may be sent on a CD or via email. Put the file and the installer within a same directory or copy it to the installation directory and complete the registration.

The proper way to register for a license file is pay to win.rar GmbH.
Yet this article will discuss the generation process of ``RARreg.key''. (Valid from version 4.x till current.)

\section{Prerequisites}
To generate the license file, one must provide his name or the company name --- referred to as ``Name''. Name containing characters other than ASCII characters is discouraged. (May lead to wrong key.)

Another information is provided by the receipt after the payment.
``License Type'' is a string that usually describes how many copies is authorized the buyer to keep.
For instance, ``Single PC usage license'', ``1000 PC usage license'', etc.

\section{A Variant of SHA-1 Algorithm}\label{SHA-1}
WinRAR adopted SHA-1 algorithm as message hash method. In general SHA-1 algorithm implementation, the digest $S$ is combined with 5 state values as
\[S=S_0S_1S_2S_3S_4\]
where $S_i$ ($i\in\left\{0,1,2,3,4\right\}$) are 32-bit unsigned integers.

The extra step before generate $S$ is doing byte-wise reverse for each $S_i$. After that concatenate them as $S$, a 160-bit unsigned integer. Note that we will store $S$ or other data discussed below in big-endian (or network order) --- the most significant byte in the front (or ``on the left'').

\subsection{Yet Another Variant}\label{YA-SHA-1}
There is another SHA-1 function used by WinRAR with all initial state values set to zero.
This function is found as the only use to generate a secret numbers with a zero-length string input, and the answer is
\[\mathtt{1050D90D0F27A54653461BD1B4E33C7C0FFD8D43}\]

\section{Digital Signature Algorithm}
The digital signature algorithm (the DSA) used by WinRAR is a variant of the SM2 digital signature algorithm.

\subsection{The Composite Field}
WinRAR chose a composite field of $\mathbb{F}_{\left(2^{15}\right)^{17}}$, described as follows.

\subsubsection{The Base Field}
The base field of $\mathbb{F}_{2^{15}}$ is generated by the primitive polynomial
\[B\left(x\right)=x^{15}+x+1\]
where the coefficients are in $\mathbb{F}_{2}$.

$\forall a\left(x\right)\in\mathbb{F}_{2^{15}}$,
\[a\left(x\right)=a_{14}x^{14}+a_{13}x^{13}+\cdots+a_1x+a_0\]
coefficients are combined as a 15-bit series, denoted as
\[a=a_{14}a_{13}\cdots a_1a_0\]

\subsubsection{The Extension Field}
The extension field of $\mathbb{F}_{\left(2^{15}\right)^{17}}$ is constructed by the primitive polynomial
\[E\left(x\right)=x^{17}+x^3+1\]
where the coefficients are in finite field $\mathbb{F}_{2^{15}}$.

$\forall b\left(x\right)\in\mathbb{F}_{\left(2^{15}\right)^{17}}$,
\[b\left(x\right)=b_{16}x^{16}+b_{15}x^{15}+\cdots+b_1x+b_0\]
and $b$ is denoted $b$ as
\[b=b_{16}b_{15}\cdots b_1b_0\]

Note that, $\forall b_i\in\mathbb{F}_{2^{15}}$, $i\in\left\{0,1,\cdots,15,16\right\}$, which means $b_i$ can be translated into a 15-bit series, and the total length of $b$ is $15\times 17=255$ bits.

\subsection{The Elliptic Curve}\label{curve}
The selected curve $C$ is
\[y^2+xy=x^3+\alpha x^2+\beta\]
where $x, y, \alpha, \beta\in\mathbb{F}_{\left(2^{15}\right)^{17}}$. WinRAR chose $\alpha=0$ and $\beta=161$, and a base point $G\in C$: (all numbers below are denoted as hexadecimal)
\begin{align*}
      G   & = \left(x_G, y_G\right)                                                     \\
      x_G & = \mathtt{56FDCBC6A27ACEE0CC2996E0096AE74FEB1ACF220A2341B898B549440297B8CC} \\
      y_G & = \mathtt{20DA32E8AFC90B7CF0E76BDE44496B4D0794054E6EA60F388682463132F931A7}
\end{align*}
And the order of $G$:
\[\mu=\mathtt{1026DD85081B82314691CED9BBEC30547840E4BF72D8B5E0D258442BBCD31}\]

\subsection{Key Generation}\label{key-gen}
WinRAR use some string (ASCII) as seed to generate private--public key pair.

\subsubsection{The Private Key}
\paragraph{First, generate hash digest for the input message.}
Calculate the digest for the input message if the length of input message is not zero, use the method described in section \ref{SHA-1}. Assign the digest to $g$.

If zero-length string is input, directly assign
\[g=\mathtt{CDE43B4C6847B9D5DC5EF4A350265329EB3EB781}\]

Now we treat $g$ a 20-byte octet stream, (significant byte first,) and concatenate a counter $c$ of a 32-bit unsigned integer after the last byte of $g$.
We will use $\parallel$ to denote ``concatenation'', which means a message $M$ should be
\[M=g\parallel c\]
which is a 25-byte stream.

\paragraph{The loop strats} by setting the counter $c$ to $1$. Send $M$ as message to the SHA-1 function described in section \ref{SHA-1}, and store the digest as $S$.
Obviously, $S$ is a 20-byte octet stream, denoted as
\[S=S_{19}S_{18}\cdots S_1S_0\]
in network order, where $S_i$ ($i\in\left\{0,1,\cdots,18,19\right\}$) are the bytes.

Assume $k$ is another octet stream, with zero-length before the loop starts.
The most least two bytes of $S$ will be taken to append to the left side of $k$:
\[k=S_1\parallel S_0\parallel k\]

Loop this process for 15 rounds, after each round increment $c$ by 1 and update to $M$.

If the digest after the $i$-time loop is denoted as $S^i$, the final $k$ after all 15 round will looks like:
\[k=S^{15}_1S^{15}_0S^{14}_1S^{14}_0\cdots S^2_1S^2_0S^1_1S^1_0\]
and this is the private key we generated.

\paragraph{To verify the key generator,} check if an empty message input generates $k=k_0$, and $k_0$ described as follows:
\[k_0=\mathtt{59FE6ABCCA90BDB95F0105271FA85FB9F11F467450C1AE9044B7FD61D65E}\]

\subsubsection{The Public Key}
As the base point $G$ on the elliptic curve is known (section \ref{curve}), the public key is calculated by multiplying the private key $k$ to the base point, according to elliptic curve arithmetics
\[P=k\cdot G\]

\paragraph{To verify the key generator,} check if $k_0$ (as described before) generates $P=P_0$, and
\begin{align*}
      P_0 & =(x_P,y_P)                                                                 \\
      x_P & =\mathtt{3861220ED9B36C9753DF09A159DFB148135D495DB3AF8373425EE9A28884BA1A} \\
      y_P & =\mathtt{12B64E62DB43A56114554B0CBD573379338CEA9124C8443C4F50E6C8B013EC20}
\end{align*}

\paragraph{A public key compress method} is used by the SM2 digital signature algorithm.
But WinRAR followed only some simplified steps:
\begin{enumerate}
      \item Let $P=\left(x_P,y_P\right)$ on the elliptic curve. if $x_P=0$, $\tilde{y}_P=0$; or else $\tilde{y}_P$ is the most least (right-most) bit of $e=y_P\cdot x_P^{-1}$ of the composite field $\mathbb{F}_{\left(2^{15}\right)^{17}}$;
      \item Concatenate $x_P$ and $\tilde{y}_P$ and obtain the bit stream.
\end{enumerate}

The compressed public key is denoted as
\[\tilde{P}=x_P\parallel\tilde{y}_P\]

\subsection{The Signing Process}\label{sign}
A message $M$ is signed using a private key $k$, following these steps:
\begin{enumerate}
      \item Pick a random number $n$, which satisfies $0<n<\mu$;\label{signing-start}
      \item Generate a digest $h$ via the algorithm described in section \ref{SHA-1}, and extend $h$ by pushing the least 10 bytes of the secret number (described in \ref{YA-SHA-1}) to its left side, which has total 30 bytes and may looks like
            \[h=\mathtt{1BD1B4E33C7C0FFD8D43}\parallel\mathrm{Sha}_1\left(M\right)\]
      \item For a point $P=\left(x_P,y_P\right)$ on the elliptic curve, let $X\left(P\right)=x_P$, then calculate
            \[r\equiv X\left(n\cdot G\right)+h\mod{\mu}\]
      \item Check if either $r=0$ or $r+n=\mu$, go to step \ref{signing-start}, else continue;
      \item Calculate
            \[s\equiv n-k\cdot r\mod{\mu}\]
      \item Check if $s=0$, go to step \ref{signing-start} or obtain signature $\left(r,s\right)$.
\end{enumerate}

\section{The Generation of the License}
Assume the input message ``Name'' and ``License Type'' are denoted as $U$ and $L$, and follow the next steps to generate the license:
\begin{enumerate}
      \item Follow the steps in section \ref{key-gen} to obtain private--public key pair with input $U$, and convert to the compressed public key form $\tilde{P}_U$;
      \item Convert $\tilde{P}_U$ into hexadecimal string form, pad with \texttt{0} on the left side until the length of the string is 64;
      \item Split the string form of $\tilde{P}_U$ into two parts --- the first 48 characters ($s^+$) and the remainders ($s^-$);
      \item Let $D_3$ be a string that is constructed as follows:
            \[D_3=\texttt{"60"}\parallel s^+\]
            where text between two double quotation mark is a string literal;
      \item Follow the steps in section \ref{key-gen} to obtain private--public key pair with input $D_3$, and convert to the compressed public key form $\tilde{P}_3$;
      \item Convert $\tilde{P}_3$ into hexadecimal string form, pad with \texttt{0} on the left side until the length of the string is 64, denoted as $D_0$;
      \item Let $I$ be a 20-character string that is constructed as follows:
            \[I=s^-\parallel D_{0,0}D_{0,1}D_{0,2}D_{0,3}\]
            where $D_{0,i}$ is the $i$th character of the string $D_0$;\\
            (\emph{Note:} In practice, WinRAR does not case the contents of $I$ at all.)
      \item Use the algorithm described in \ref{sign} with $L$ as message input and $k_0$ as private key input, obtain signature $\left(r_L,s_L\right)$;
      \item Convert $r_L$ and $s_L$ into hexadecimal string form, $s_L^+$ and $s_L^-$, pad each with \texttt{0} on its left side until both length are 60; (Remain unchanged if the length is greater than 60)
      \item Let $D_1$ be a string that is constructed as follows:
            \[D_1=\texttt{"60"}\parallel s_L^-\parallel s_L^+\]
      \item Construct a message string $M_1$ as
            \[M_1=U\parallel D_0\]
            and sign it with private key $k_0$, obtain $\left(r_1,s_1\right)$;
      \item Convert $r_1$ and $s_1$ into hexadecimal string form, $s_1^+$ and $s_1^-$, pad each with \texttt{0} on its left side until both length are 60; (Remain unchanged if the length is greater than 60)
      \item Construct a string $D_2$ as
            \[D_2=\texttt{"60"}\parallel s_1^-\parallel s_1^+\]
      \item A CRC32 checksum is calculated using the message string
            \[L\parallel U\parallel D_0\parallel D_1\parallel D_2\parallel D_3\]
      \item Convert the checksum into \textbf{decimal} string form $s_c$, pad with \texttt{0} on the left side until the length is 10;
      \item Let $l_0$, $l_1$, $l_2$ and $l_3$ are the length of $D_0$, $D_1$, $D_2$ and $D_3$, respectively, and convert $l_i$ into decimal string forms $s_{l,i}$;
      \item Let $D$ be a string constructed as follows:
            \[D=s_{l,0}\parallel s_{l,1}\parallel s_{l,2}\parallel s_{l,3}\parallel D_0\parallel D_1\parallel D_2\parallel D_3 \parallel s_c\]
\end{enumerate}

\subsection{Output}
\textbf{The first line:} a string literal: \texttt{"RAR registration data"}.\\
\textbf{The second line:} $U$.\\
\textbf{The third line:} $L$.\\
\textbf{The fourth line:} combined with a string literal \texttt{"UID="} and $I$.\\
\textbf{The following lines:} $D$ separated into 7 lines, while the first 6 lines have 54 characters each.

\section*{Reference}
\begin{enumerate}
      \item \url{https://en.wikipedia.org/wiki/WinRAR}
      \item \url{https://github.com/bitcookies/winrar-keygen}
      \item \url{https://github.com/obaby/winrar-keygen}
      \item ``GB/T 32918—2016: Information security technology—Public key cryptographic algorithm SM2 based on elliptic curves''
\end{enumerate}

\end{document}